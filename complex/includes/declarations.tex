%----------------------------------------------------------------------------------------
%	COLOR DEFINITIONS
%----------------------------------------------------------------------------------------

% we use the xtended color package
\usepackage[table,usenames,dvipsnames]{xcolor}

% citation reference color
\definecolor{CITATION_COL}{RGB}{121,0,226}

% urls color
\definecolor{SECTION_COL}{RGB}{0, 117,226}

% internal references color
\definecolor{LINK_COL}{rgb}{0,0,0}

% main color
\definecolor{MAIN_COL}{RGB}{0, 117,226}
% code highlight color
\definecolor{CODE_COL}{rgb}{0.8,0.8,0.8}

% some black fallback
\definecolor{DEF_COL}{RGB}{226,109,0}

%----------------------------------------------------------------------------------------
%	REFERENCES URL CITATION STYLES AND SO ON
%----------------------------------------------------------------------------------------

% apply color definitions to hyperref package
\usepackage[colorlinks=true,citecolor=CITATION_COL,urlcolor=SECTION_COL,linkcolor=LINK_COL]{hyperref}

% make urls directly open the browser onclick
\usepackage{url}
\urlstyle{same}


%----------------------------------------------------------------------------------------
%	FONT DEFINITIONS
%----------------------------------------------------------------------------------------

% better typography rendering scheme
\usepackage[protrusion=true,expansion=true]{microtype} 

 % we use the standard Times font
\usepackage{pslatex}

 % required for accented characters
\usepackage[T1]{fontenc}

% change line spacing here, readability 
% benefits from a slight increase by default
\linespread{1.25} 

% allows more font size definitions
\usepackage{moresize}

%----------------------------------------------------------------------------------------
%	GEOMETRY  DEFINITIONS
%----------------------------------------------------------------------------------------

% define page styles using geometry
\usepackage[bottom=3cm,marginparwidth=1.5cm, marginparsep=0.5cm]{geometry}

 %define A4 Paper
\geometry{a4paper}	

% define margins for even and odd pages
\setlength{\oddsidemargin}{15.5pt}
\setlength{\evensidemargin}{15.5pt}

% paragraph indent and skip
\setlength{\parindent}{0mm}
\setlength{\parskip}{1mm }

% short command for colorizing a text as a note
% not used within this thesis anymore
\newcommand{\NOTE}[1]{\textit{\textcolor{DEF_COL}{#1}}}
	%\marginpar{\sffamily \MakeUppercase{\tiny{{#1}}}}


%----------------------------------------------------------------------------------------
%	GRAPHICS  DEFINITIONS
%----------------------------------------------------------------------------------------

% required for including pictures
\usepackage[pdftex]{graphicx}

 % make it possible to include more than one captioned figure/table in a single float 
\usepackage{subfig}

% make it possible to control text
% positioning around figures
\usepackage{wrapfig}

 % for much better looking tables
\usepackage{booktabs}	

% for better arrays (eg matrices) in maths		
\usepackage{array} 


%----------------------------------------------------------------------------------------
%	ENVIRONMENT  DEFINITIONS
%----------------------------------------------------------------------------------------

 % very flexible & customisable lists
%(eg. enumerate/itemize, etc.)				
\usepackage{paralist}				

% adds environment for commenting
% out blocks of text & for better verbatim
\usepackage{verbatim}				

% allows multiple rows to be merged to one row
\usepackage{multirow}

%----------------------------------------------------------------------------------------
% CUSTOM STRUT FOR EMPTY BOXES
%----------------------------------------- -----------------------------------------------

% when boxes become a pain this might help
\newcommand{\mystrut}{\rule[-.3\baselineskip]{0pt}{\baselineskip}}

%----------------------------------------------------------------------------------------
%	HEADER / FOOTER  DEFINITIONS
%----------------------------------------------------------------------------------------

% header definitions package
 % this should be set AFTER setting up the page geometry
\usepackage{fancyhdr}

% we want to shorten the display of the chapter in the header
\renewcommand{\chaptermark}[1]
{
  \markboth{#1}{}
}

% but we want to show extended info of the current section
\renewcommand{\sectionmark}[1]{\markright{\thesection\ #1}}

% we can even manipulate the font family
%\lhead{\sffamily Chapter \thechapter}

% this defines, what content should be displayed in header and footer
% and on which side (left-right) as well as on even and odd pages
\fancyhf[FLE,FRO]{}
\fancyhf[HLE,HRO]{\colorbox{MAIN_COL}{\sffamily\LARGE\textcolor{white}{\thepage}}}
\fancyhf[HRE,HLO]{\sffamily \textcolor{MAIN_COL}{\leftmark}}
\cfoot{}

%override plain page style for plain pages
\fancypagestyle{plain}{ %
  \fancyhf{} % remove everything
  \renewcommand{\headrulewidth}{0pt} % remove lines as well
  \renewcommand{\footrulewidth}{0pt}
  \fancyhf[HLE,HRO]{\colorbox{MAIN_COL}{\sffamily\large\textcolor{white}{\thepage}}}
  \cfoot{}
}

%----------------------------------------------------------------------------------------
%	SECTION TITLE APPEARANCE
%----------------------------------------------------------------------------------------

% we want to customize our section title 
% to make it look more appealing
\usepackage{sectsty}

% for dimensioning and other math stuff
\usepackage{amsmath}

% modify bg colors
\usepackage[some]{background}

% draw custom graphics progammatically
\usepackage{tikz}				
\usetikzlibrary{shapes, backgrounds,mindmap, trees}

% this is our custom section page layout
\newcommand{\SECTBOX}[1]
{
	\begin{tikzpicture}[remember picture,overlay]
 		\path [fill=MAIN_COL] (current page.west)rectangle (current page.north east); 
	\end{tikzpicture}
}

% this is our custom title page layout
\newcommand{\TITLEBOX}
{
	\begin{tikzpicture}[remember picture,overlay]
 		\path [fill=MAIN_COL] (current page.north west) rectangle (25.0,-3.5); 
	\end{tikzpicture}
}


 % (See the fntguide.pdf for font help)
\allsectionsfont{\sffamily\mdseries\upshape}

% for customizing the title page
\usepackage[explicit]{titlesec}

% font and color def for section
\titleformat{\section}{\LARGE\sffamily}{\textcolor{LINK_COL}{\thesection \hspace{12pt} #1}}{0pt}{}

% font and color def for subsection
\titleformat{\subsection}{\large\bf\sffamily}{\textcolor{SECTION_COL}{\thesubsection \hspace{12pt} #1}}{0pt}{}

% note: remove \thesection or \thesubsection to leave out the 
% numbering at the start of the section / subsection headline


%----------------------------------------------------------------------------------------
%	TABLE OF CONTENT APPEARANCE
%----------------------------------------------------------------------------------------

% deprecated
%\usepackage[nottoc,notlof,notlot]{tocbibind} % Put the bibliography in the ToC

% Alter the style of the Table of Contents
%\usepackage[titles,subfigure]{tocloft}

%\renewcommand{\cftsecfont}{\rmfamily\mdseries\upshape}

%\renewcommand{\cftsecpagefont}{\rmfamily\mdseries\upshape} % No bold!

% renames abstract to executive summary
\renewcommand{\abstractname}{Executive Summary} 		


%----------------------------------------------------------------------------------------
%	CUSTOM HRULE
%----------------------------------------------------------------------------------------

% a custom horizontal rule / line
\newcommand{\HRule}{\rule{\linewidth}{0.5mm}}

\newcommand{\ThinHRule}{\textcolor{CODE_COL}{\rule{0.5\linewidth}{0.05mm}}}


%----------------------------------------------------------------------------------------
%	ENVIRONMENTS
%----------------------------------------------------------------------------------------

% extended listin options
% can be used for code display
\usepackage{listings}

% syntax highlight def for listings
\lstset{ %
  	backgroundcolor=\color{CODE_COL},
	language=C,
	breaklines=true
}

% allows us to define pseudocode algorithms
\usepackage[ruled,linesnumbered]{algorithm2e}

% extended table definitions
\usepackage{tabularx}

% include external pdf pages
\usepackage{pdfpages}
\usepackage{float}


%----------------------------------------------------------------------------------------
%	Code Syntax Highlight Formatting
%----------------------------------------------------------------------------------------

\lstset{
  basicstyle=\ttfamily,
  columns=fullflexible,
  showstringspaces=false,
  commentstyle=\color{gray}\upshape
}

\lstdefinelanguage{XML}
{
  morestring=[b]",
  morestring=[s]{>}{<},
  morecomment= [s]{<!}{->},
  stringstyle=\color{black},
  identifierstyle=\color{blue},
  keywordstyle=\color{orange},
  morekeywords={xmlns, pattern,type, cells, count, required} % list your attributes here which may be highlighted
}



